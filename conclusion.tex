\section*{Conclusion}

In response to the inherent limitations of traditional Constant Function Automated Market Makers---namely Impermanent Loss, price impact, and capital inefficiency---this paper has introduced the \textbf{Trading-Weighted Market Maker (TWMM)}. This novel architecture presents a paradigm shift, treating liquidity provision not as passive price discovery but as a form of active portfolio management.

\subsection*{A Paradigm Shift in Liquidity Management}

The TWMM re-conceptualizes liquidity provision as \textbf{active portfolio management}. By utilizing oracle-based pricing and establishing \textbf{target shares} for each asset, it decouples trade execution from the pool's liquidity depth, eliminating the price impact typically associated with large trades. The core of the model is its dynamic, deviation-based fee and cashback system. This powerful incentive structure continuously encourages market participants to arbitrage the pool back toward its target equilibrium, creating a self-regulating and stable system.

\subsection*{Superior Capital Efficiency and Risk Mitigation}

The practical results of this architecture are significant. Our analysis demonstrates that the TWMM can achieve a level of trade execution performance comparable to a CF-AMM with \textbf{tenfold the liquidity}, representing a profound improvement in capital efficiency. Furthermore, the integrated mechanisms of deviation limits and dynamic fees provide robust protection against oracle price risk and help mitigate impermanent loss, transforming the LP token into a representation of a managed, risk-adjusted strategy.

\subsection*{Future Directions}

The TWMM framework serves as a robust foundation for a new generation of decentralized financial instruments. Future research may explore more sophisticated non-linear fee functions, deeper integration with MEV-aware oracle solutions, and expanded applications in areas like treasury management and structured products. As the decentralized finance landscape matures, such robust and capital-efficient models will be essential in building the next generation of financial infrastructure. 

% ORACLE dependency here