\section*{Conclusion}
The TWMM is a powerful system that successfully challenges current issues with on-chain decentralized liquidity provision. The problem of impermanent loss is addressed through the ability to actively manage asset shares, causing them to rebalance. This creates a new type of on-chain derivative that represents a portfolio strategy. 

The deviation fee model provides strong incentives to keep asset shares aligned with their target values. Because the TWMM supports multiple assets in one pool, it also increases trading volume capture from liquidity provision, which can be further optimized by adjusting asset weights to provide deeper liquidity where needed. 

The LVR problem is addressed as the TWMM uses frequently updated oracle prices, and the pool shows significant advantages in capital efficiency over CF-AMMs.

The model has a potential bottleneck caused by oracle price inefficiencies. If oracle prices have errors that overcome the collected fee value, an opportunity to drain the pool by extracting arbitrage profit can appear (same mechanic as LVR). This, however, is more complicated if the pool is nearly balanced, as additional volume would also involve deviation fees that grow proportionally with the trade size and would counteract the profitability of exploiting the oracle error. Therefore, effective oracles are key to the profitability of the TWMM.