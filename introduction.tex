\section{Introduction}

At the time of this paper's writing, on-chain liquidity provision predominantly relies on \textbf{Constant Function Automated Market Makers (CF-AMMs)}. The evolution of CF-AMMs led to the concept of \textbf{concentrated liquidity}, where liquidity is supplied only within a specific price range. This innovation significantly improves capital efficiency and offers enhanced tools for managing the risks associated with Impermanent Loss (IL) and Loss-Versus-Rebalancing (LVR), as comprehensively detailed in research from a16z and Columbia University\cite{AMMandLVR}. 

Despite these advancements, market analysis indicates that even concentrated liquidity AMMs (CL-AMMs) face substantial risks and present considerable management complexity. Statistics from CrocSwap \cite{CrocSwap2023} reveal that only half of liquidity providers in the ETH/USDC trading pair—currently the highest volume pair—experience positive returns. Furthermore, the majority of Profit and Loss (PnL) outcomes are statistically clustered around zero, rendering liquidity provision a challenging and high-risk endeavor.

Alternative solutions are emerging to optimize LVR and IL. These include approaches that combine Request for Quote (RFQ) mechanisms with AMMs, such as those explored by Maverick Protocol \cite{MaverickAMM2023}, or methods that minimize harmful trade impacts through trade batching, as implemented by CoW Swap pools \cite{Cowswap2025}.

Another critical challenge in liquidity provision is the prevalence of paired on-chain liquidity, which restricts volume acquisition and forces Liquidity Providers (LPs) to select optimal trading pairs. Projects like Balancer \cite{Balancer2019} and Paradigm's innovative approach \cite{Orbital2025} (an advanced iteration of Curve Finance's 3-pool) address this by introducing multi-dimensional curves. These innovations enable the creation of pools that facilitate simultaneous trades across a basket of assets. Conceptually, they transition from a simple curve of $f(x,y) = L$ to a more generalized form of $f(\mathbf{X}) = L$, where $\mathbf{X} = [x_1, x_2, \dots, x_i]$ represents a vector of $i$ assets.

The \textbf{Target Weighted Market Maker (TWMM)} system introduces an innovative approach to on-chain market making, designed specifically to address these aforementioned challenges.